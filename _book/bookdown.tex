\documentclass[]{ctexbook}
\usepackage{lmodern}
\usepackage{amssymb,amsmath}
\usepackage{ifxetex,ifluatex}
\usepackage{fixltx2e} % provides \textsubscript
\ifnum 0\ifxetex 1\fi\ifluatex 1\fi=0 % if pdftex
  \usepackage[T1]{fontenc}
  \usepackage[utf8]{inputenc}
\else % if luatex or xelatex
  \ifxetex
    \usepackage{xltxtra,xunicode}
  \else
    \usepackage{fontspec}
  \fi
  \defaultfontfeatures{Ligatures=TeX,Scale=MatchLowercase}
\fi
% use upquote if available, for straight quotes in verbatim environments
\IfFileExists{upquote.sty}{\usepackage{upquote}}{}
% use microtype if available
\IfFileExists{microtype.sty}{%
\usepackage{microtype}
\UseMicrotypeSet[protrusion]{basicmath} % disable protrusion for tt fonts
}{}
\usepackage[b5paper,tmargin=2.5cm,bmargin=2.5cm,lmargin=3.5cm,rmargin=2.5cm]{geometry}
\usepackage[unicode=true]{hyperref}
\PassOptionsToPackage{usenames,dvipsnames}{color} % color is loaded by hyperref
\hypersetup{
            pdftitle={数据操纵手册},
            pdfauthor={张晨阳},
            colorlinks=true,
            linkcolor=Maroon,
            citecolor=Blue,
            urlcolor=Blue,
            breaklinks=true}
\urlstyle{same}  % don't use monospace font for urls
\usepackage{natbib}
\bibliographystyle{apalike}
\usepackage{color}
\usepackage{fancyvrb}
\newcommand{\VerbBar}{|}
\newcommand{\VERB}{\Verb[commandchars=\\\{\}]}
\DefineVerbatimEnvironment{Highlighting}{Verbatim}{commandchars=\\\{\}}
% Add ',fontsize=\small' for more characters per line
\usepackage{framed}
\definecolor{shadecolor}{RGB}{248,248,248}
\newenvironment{Shaded}{\begin{snugshade}}{\end{snugshade}}
\newcommand{\AlertTok}[1]{\textcolor[rgb]{0.94,0.16,0.16}{#1}}
\newcommand{\AnnotationTok}[1]{\textcolor[rgb]{0.56,0.35,0.01}{\textbf{\textit{#1}}}}
\newcommand{\AttributeTok}[1]{\textcolor[rgb]{0.77,0.63,0.00}{#1}}
\newcommand{\BaseNTok}[1]{\textcolor[rgb]{0.00,0.00,0.81}{#1}}
\newcommand{\BuiltInTok}[1]{#1}
\newcommand{\CharTok}[1]{\textcolor[rgb]{0.31,0.60,0.02}{#1}}
\newcommand{\CommentTok}[1]{\textcolor[rgb]{0.56,0.35,0.01}{\textit{#1}}}
\newcommand{\CommentVarTok}[1]{\textcolor[rgb]{0.56,0.35,0.01}{\textbf{\textit{#1}}}}
\newcommand{\ConstantTok}[1]{\textcolor[rgb]{0.00,0.00,0.00}{#1}}
\newcommand{\ControlFlowTok}[1]{\textcolor[rgb]{0.13,0.29,0.53}{\textbf{#1}}}
\newcommand{\DataTypeTok}[1]{\textcolor[rgb]{0.13,0.29,0.53}{#1}}
\newcommand{\DecValTok}[1]{\textcolor[rgb]{0.00,0.00,0.81}{#1}}
\newcommand{\DocumentationTok}[1]{\textcolor[rgb]{0.56,0.35,0.01}{\textbf{\textit{#1}}}}
\newcommand{\ErrorTok}[1]{\textcolor[rgb]{0.64,0.00,0.00}{\textbf{#1}}}
\newcommand{\ExtensionTok}[1]{#1}
\newcommand{\FloatTok}[1]{\textcolor[rgb]{0.00,0.00,0.81}{#1}}
\newcommand{\FunctionTok}[1]{\textcolor[rgb]{0.00,0.00,0.00}{#1}}
\newcommand{\ImportTok}[1]{#1}
\newcommand{\InformationTok}[1]{\textcolor[rgb]{0.56,0.35,0.01}{\textbf{\textit{#1}}}}
\newcommand{\KeywordTok}[1]{\textcolor[rgb]{0.13,0.29,0.53}{\textbf{#1}}}
\newcommand{\NormalTok}[1]{#1}
\newcommand{\OperatorTok}[1]{\textcolor[rgb]{0.81,0.36,0.00}{\textbf{#1}}}
\newcommand{\OtherTok}[1]{\textcolor[rgb]{0.56,0.35,0.01}{#1}}
\newcommand{\PreprocessorTok}[1]{\textcolor[rgb]{0.56,0.35,0.01}{\textit{#1}}}
\newcommand{\RegionMarkerTok}[1]{#1}
\newcommand{\SpecialCharTok}[1]{\textcolor[rgb]{0.00,0.00,0.00}{#1}}
\newcommand{\SpecialStringTok}[1]{\textcolor[rgb]{0.31,0.60,0.02}{#1}}
\newcommand{\StringTok}[1]{\textcolor[rgb]{0.31,0.60,0.02}{#1}}
\newcommand{\VariableTok}[1]{\textcolor[rgb]{0.00,0.00,0.00}{#1}}
\newcommand{\VerbatimStringTok}[1]{\textcolor[rgb]{0.31,0.60,0.02}{#1}}
\newcommand{\WarningTok}[1]{\textcolor[rgb]{0.56,0.35,0.01}{\textbf{\textit{#1}}}}
\usepackage{longtable,booktabs}
% Fix footnotes in tables (requires footnote package)
\IfFileExists{footnote.sty}{\usepackage{footnote}\makesavenoteenv{long table}}{}
\IfFileExists{parskip.sty}{%
\usepackage{parskip}
}{% else
\setlength{\parindent}{0pt}
\setlength{\parskip}{6pt plus 2pt minus 1pt}
}
\setlength{\emergencystretch}{3em}  % prevent overfull lines
\providecommand{\tightlist}{%
  \setlength{\itemsep}{0pt}\setlength{\parskip}{0pt}}
\setcounter{secnumdepth}{5}
% Redefines (sub)paragraphs to behave more like sections
\ifx\paragraph\undefined\else
\let\oldparagraph\paragraph
\renewcommand{\paragraph}[1]{\oldparagraph{#1}\mbox{}}
\fi
\ifx\subparagraph\undefined\else
\let\oldsubparagraph\subparagraph
\renewcommand{\subparagraph}[1]{\oldsubparagraph{#1}\mbox{}}
\fi

% set default figure placement to htbp
\makeatletter
\def\fps@figure{htbp}
\makeatother

\usepackage{booktabs}
\usepackage{longtable}

\usepackage{framed,color}
\definecolor{shadecolor}{RGB}{248,248,248}

\renewcommand{\textfraction}{0.05}
\renewcommand{\topfraction}{0.8}
\renewcommand{\bottomfraction}{0.8}
\renewcommand{\floatpagefraction}{0.75}

\let\oldhref\href
\renewcommand{\href}[2]{#2\footnote{\url{#1}}}

\makeatletter
\newenvironment{kframe}{%
\medskip{}
\setlength{\fboxsep}{.8em}
 \def\at@end@of@kframe{}%
 \ifinner\ifhmode%
  \def\at@end@of@kframe{\end{minipage}}%
  \begin{minipage}{\columnwidth}%
 \fi\fi%
 \def\FrameCommand##1{\hskip\@totalleftmargin \hskip-\fboxsep
 \colorbox{shadecolor}{##1}\hskip-\fboxsep
     % There is no \\@totalrightmargin, so:
     \hskip-\linewidth \hskip-\@totalleftmargin \hskip\columnwidth}%
 \MakeFramed {\advance\hsize-\width
   \@totalleftmargin\z@ \linewidth\hsize
   \@setminipage}}%
 {\par\unskip\endMakeFramed%
 \at@end@of@kframe}
\makeatother

\makeatletter
\@ifundefined{Shaded}{
}{\renewenvironment{Shaded}{\begin{kframe}}{\end{kframe}}}
\@ifpackageloaded{fanyverb}{%
  % https://github.com/CTeX-org/ctex-kit/issues/331
  \RecustomVerbatimEnvironment{Highlighting}{Verbatim}{commandchars=\\\{\},formatcom=\xeCJKVerbAddon}%
}{}
\makeatother

\usepackage{makeidx}
\makeindex

\urlstyle{tt}

\usepackage{amsthm}
\makeatletter
\def\thm@space@setup{%
  \thm@preskip=8pt plus 2pt minus 4pt
  \thm@postskip=\thm@preskip
}
\makeatother

\frontmatter

\title{数据操纵手册}
\author{张晨阳}
\date{2018-06-27}

\let\BeginKnitrBlock\begin \let\EndKnitrBlock\end
\begin{document}
\maketitle


\thispagestyle{empty}

\begin{center}
献给……

呃,爱谁谁吧
\end{center}

\setlength{\abovedisplayskip}{-5pt}
\setlength{\abovedisplayshortskip}{-5pt}

{
\setcounter{tocdepth}{2}
\tableofcontents
}
\listoftables
\listoffigures
\chapter*{前言}


一些对过去工作经验零散凌乱的总结让我有了将其汇总成书的念头。又介于自己并非是什么成功人士、学霸大佬或者文笔绚烂的文人,所以迟迟未``动笔'',总觉得这并非是我该做的事情。

由于 \textbf{bookdown}\index{bookdown} \citep{R-bookdown} 包给 R
语言用户提供了一种便捷的写作方式,这让我有了尝试的动力。

总的来说这本的书是关于如何操作数据的。由于
\textbf{kintr}\index{kintr}\citep{xie2015}
强大的多语言支持,可以让我不局限于 R 语言,同样深受欢迎的 Python 和
SQL。

\begin{Shaded}
\begin{Highlighting}[]
\KeywordTok{sessionInfo}\NormalTok{()}
\end{Highlighting}
\end{Shaded}

\begin{verbatim}
## R version 3.5.0 (2018-04-23)
## Platform: x86_64-apple-darwin15.6.0 (64-bit)
## Running under: macOS High Sierra 10.13
## 
## Matrix products: default
## BLAS: /Library/Frameworks/R.framework/Versions/3.5/Resources/lib/libRblas.0.dylib
## LAPACK: /Library/Frameworks/R.framework/Versions/3.5/Resources/lib/libRlapack.dylib
## 
## locale:
## [1] en_US.UTF-8/en_US.UTF-8/en_US.UTF-8/C/en_US.UTF-8/en_US.UTF-8
## 
## attached base packages:
## [1] stats     graphics  grDevices utils     datasets 
## [6] methods   base     
## 
## other attached packages:
## [1] reticulate_1.7
## 
## loaded via a namespace (and not attached):
##  [1] Rcpp_0.12.17    bookdown_0.7    lattice_0.20-35
##  [4] digest_0.6.15   rprojroot_1.3-2 grid_3.5.0     
##  [7] DBI_1.0.0       jsonlite_1.5    backports_1.1.2
## [10] magrittr_1.5    evaluate_0.10.1 highr_0.6      
## [13] stringi_1.2.2   rstudioapi_0.7  Matrix_1.2-14  
## [16] rmarkdown_1.9   RMySQL_0.10.15  tools_3.5.0    
## [19] stringr_1.3.1   xfun_0.1        yaml_2.1.19    
## [22] compiler_3.5.0  htmltools_0.3.6 knitr_1.20
\end{verbatim}

\section*{致谢}


感谢我的父母吧!

\BeginKnitrBlock{flushright}
张晨阳\\
上海
\EndKnitrBlock{flushright}

\hypertarget{author}{%
\chapter*{作者简介}\label{author}}


徜徉在数据的中的一个默默无闻的清道夫。

\mainmatter

\hypertarget{intro}{%
\chapter{牛刀小试}\label{intro}}

现在我们可以试试 \textbf{bookdown} 的一些初级功能了,例如图表。图
\ref{fig:hello} 是一幅无趣的散点图,表 \ref{tab:iris} 是一份枯燥的数据。

\begin{Shaded}
\begin{Highlighting}[]
\KeywordTok{head}\NormalTok{(mtcars, }\DecValTok{5}\NormalTok{)}
\end{Highlighting}
\end{Shaded}

\begin{verbatim}
##                    mpg cyl disp  hp drat    wt  qsec
## Mazda RX4         21.0   6  160 110 3.90 2.620 16.46
## Mazda RX4 Wag     21.0   6  160 110 3.90 2.875 17.02
## Datsun 710        22.8   4  108  93 3.85 2.320 18.61
## Hornet 4 Drive    21.4   6  258 110 3.08 3.215 19.44
## Hornet Sportabout 18.7   8  360 175 3.15 3.440 17.02
##                   vs am gear carb
## Mazda RX4          0  1    4    4
## Mazda RX4 Wag      0  1    4    4
## Datsun 710         1  1    4    1
## Hornet 4 Drive     1  0    3    1
## Hornet Sportabout  0  0    3    2
\end{verbatim}

\begin{Shaded}
\begin{Highlighting}[]
\ImportTok{import}\NormalTok{ pandas}
\BuiltInTok{print}\NormalTok{(r.mtcars.head())}
\end{Highlighting}
\end{Shaded}

\begin{verbatim}
##                     mpg  cyl   disp     hp  drat     wt   qsec   vs   am  \
## Mazda RX4          21.0  6.0  160.0  110.0  3.90  2.620  16.46  0.0  1.0   
## Mazda RX4 Wag      21.0  6.0  160.0  110.0  3.90  2.875  17.02  0.0  1.0   
## Datsun 710         22.8  4.0  108.0   93.0  3.85  2.320  18.61  1.0  1.0   
## Hornet 4 Drive     21.4  6.0  258.0  110.0  3.08  3.215  19.44  1.0  0.0   
## Hornet Sportabout  18.7  8.0  360.0  175.0  3.15  3.440  17.02  0.0  0.0   
## 
##                    gear  carb  
## Mazda RX4           4.0   4.0  
## Mazda RX4 Wag       4.0   4.0  
## Datsun 710          4.0   1.0  
## Hornet 4 Drive      3.0   1.0  
## Hornet Sportabout   3.0   2.0
\end{verbatim}

\begin{Shaded}
\begin{Highlighting}[]
\KeywordTok{select}\NormalTok{ * }\KeywordTok{from}\NormalTok{ mtcars }\KeywordTok{limit} \DecValTok{5}
\end{Highlighting}
\end{Shaded}

\begin{table}

\caption{\label{tab:mysql}5 records}
\centering
\begin{tabular}[t]{l|r|r|r|r|r|r|r|r|r|r|r}
\hline
rownames & mpg & cyl & disp & hp & drat & wt & qsec & vs & am & gear & carb\\
\hline
Mazda RX4 & 21.0 & 6 & 160 & 110 & 3.90 & 2.620 & 16.46 & 0 & 1 & 4 & 4\\
\hline
Mazda RX4 Wag & 21.0 & 6 & 160 & 110 & 3.90 & 2.875 & 17.02 & 0 & 1 & 4 & 4\\
\hline
Datsun 710 & 22.8 & 4 & 108 & 93 & 3.85 & 2.320 & 18.61 & 1 & 1 & 4 & 1\\
\hline
Hornet 4 Drive & 21.4 & 6 & 258 & 110 & 3.08 & 3.215 & 19.44 & 1 & 0 & 3 & 1\\
\hline
Hornet Sportabout & 18.7 & 8 & 360 & 175 & 3.15 & 3.440 & 17.02 & 0 & 0 & 3 & 2\\
\hline
\end{tabular}
\end{table}

\begin{Shaded}
\begin{Highlighting}[]
\KeywordTok{par}\NormalTok{(}\DataTypeTok{mar =} \KeywordTok{c}\NormalTok{(}\DecValTok{4}\NormalTok{, }\DecValTok{4}\NormalTok{, }\DecValTok{1}\NormalTok{, }\FloatTok{.1}\NormalTok{))}
\KeywordTok{plot}\NormalTok{(cars, }\DataTypeTok{pch =} \DecValTok{19}\NormalTok{)}
\end{Highlighting}
\end{Shaded}

\begin{figure}
\includegraphics[width=0.9\linewidth]{bookdown_files/figure-latex/hello-1} \caption{雷猴啊,散点图!}\label{fig:hello}
\end{figure}

\begin{Shaded}
\begin{Highlighting}[]
\NormalTok{knitr}\OperatorTok{::}\KeywordTok{kable}\NormalTok{(}
  \KeywordTok{head}\NormalTok{(iris), }\DataTypeTok{caption =} \StringTok{'雷猴啊,iris 数据!'}\NormalTok{,}
  \DataTypeTok{booktabs =} \OtherTok{TRUE}
\NormalTok{)}
\end{Highlighting}
\end{Shaded}

\begin{table}

\caption{\label{tab:iris}雷猴啊,iris 数据!}
\centering
\begin{tabular}[t]{rrrrl}
\toprule
Sepal.Length & Sepal.Width & Petal.Length & Petal.Width & Species\\
\midrule
5.1 & 3.5 & 1.4 & 0.2 & setosa\\
4.9 & 3.0 & 1.4 & 0.2 & setosa\\
4.7 & 3.2 & 1.3 & 0.2 & setosa\\
4.6 & 3.1 & 1.5 & 0.2 & setosa\\
5.0 & 3.6 & 1.4 & 0.2 & setosa\\
5.4 & 3.9 & 1.7 & 0.4 & setosa\\
\bottomrule
\end{tabular}
\end{table}

就这样,你可以一直编下去,直到编不下去。

\hypertarget{wind}{%
\chapter{白苹风末}\label{wind}}

瞎扯几句。

\section{张老爷子}

话说张老爷子写了一首诗:

\begin{quote}
姑苏开遍碧桃时,邂逅河阳女画师。\\
红豆江南留梦影,白苹风末唱秋词。
\end{quote}

\section{彭大将领}

貌似大家都喜欢用白萍风这个意境。又如彭玉麟的对联:

\begin{quote}
凭栏看云影波光,最好是红蓼花疏、白苹秋老;\\
把酒对琼楼玉宇,莫孤负天心月到、水面风来。
\end{quote}

嘿,玛尼玛尼哄。

\cleardoublepage

\hypertarget{appendix-}{%
\appendix \addcontentsline{toc}{chapter}{\appendixname}}


\hypertarget{sound}{%
\chapter{余音绕梁}\label{sound}}

呐,到这里朕的书差不多写完了,但还有几句话要交待,所以开个附录,再啰嗦几句,各位客官稍安勿躁、扶稳坐好。

\bibliography{book.bib,packages.bib}

\backmatter
\printindex

\end{document}
